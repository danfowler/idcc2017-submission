\documentclass{idcc}
\usepackage[backend=bibtex]{biblatex}
\title{Frictionless Data: Making Research Data Quality Visible}
\author{Dan Fowler}
\affil{Open Knowledge International}
\correspondence{Dan Fowler <daniel.fowler@okfn.org>}
\addbibresource{submission.bib}
\begin{document}
\maketitle

There is significant friction in the acquisition, sharing, and reuse
of research data.  It is estimated that eighty percent of data
analysis is invested in the cleaning and mapping of data
\cite{dasujohnson2003}.  This friction hampers researchers not well
versed in data processing techniques from reusing an ever-increasing
amount of research data available on the web and within scientific
data repositories.  Frictionless Data is an ongoing project at Open
Knowledge International focused on removing the friction in working
with data. We are doing this by developing a set of tools,
specifications, and best practices for describing, publishing, and
validating data. The heart of this project is ``Data Package'', a
containerization format for data based on existing practices for
publishing open-source software.  This Practice Paper will report on
current progress toward that goal.

\section{Tabular Data Packages}
Data Package is a format for storing useful metadata alongside a given
dataset in a simple JSON file called ``datapackage.json''.  Our
current efforts focus on packaging the very common ``tabular'' type of
data, for example, data naturally stored in CSV files.  This is a
clear area for improvement well illustrated by data guidelines
recently issued by Wellcome Open Research.  The guidelines mandated
the following:

\begin{quote}
Spreadsheets should be submitted in CSV or TAB format; EXCEPT if the
spreadsheet contains variable labels, code labels, or defined missing
values, as these should be submitted in SAV, SAS or POR format, with
the variable defined in English \cite{wellcomeopenresearch2016}.
\end{quote}

Typically, data management plans mandate that researchers submit data
in non-proprietary formats, like CSV, to ensure their long-term
accessibility.  SPSS, SAS, and other proprietary data analysis
platforms are accepted as file formats because they provide features
that, until recently, haven't been supported by a standard,
non-proprietary analog.  We believe a decentralized, open standard for
publishing tabular research data based on an existing formats like CSV
is critical to high fidelity data transport and preservation, and our
experiences so far point to an unmet need for exactly this kind of
approach.

Our Data Package specifications further provide an extensible way to
create and assign functional ``schemas'' for tabular data which define
expected types, constraints (e.g. maximum and minimum values for
columns), and relations between columns in an standard, open
format. Importantly, these specifications require few or no changes to
existing data. In this way, our approach enables further benefits: by
describing data with type and constraint information in a
machine-readable manner, datasets can be automatically validated for
adherence to the simple rules defined by the researcher or repository.

\section{Uptake}
Having started work on this concept through the process of developing
CKAN, OpenSpending, and other data-intensive civic technology
projects, we are currently in the middle of a a project to trial this
approach across various research domains.  Over the last year, we have
noticed a very positive reaction driven, we believe, by a relentless
focus on keeping the standards as simple as possible to make it easy
to develop useful tools.

In some disciplines, Data Packages can provide the foundation for a
general framework for data publishing where none previously existed.
While many research disciplines have existing standards for sharing
data, some do not.  For instance, we are working the ``Open
Archaeology'' working group to define some basic standards for
describing the type of data---a mix of tabular and geospatial
data---typically generated in that discipline.  In other disciplines,
we have seen interest even where existing standards already exist.
For instance, ecologists will typically share data using the
Ecological Metadata Language \cite{eml2016}.  However, we have
identified a software project for distributing ecological datasets
that is using the tabular data package format due in large part to the
simplicity of implementation.  We will be interviewing the team to
further elaborate on their motivations for adoption.

\section{Driving Data Quality}

Through this approach, we expect broad-based improvements in data
quality as well as increased re-use of data.  Significant time and
energy is currently lost to cleaning data by early career researchers,
many of whom may be more interested in generating novel insights than
the sometimes tedious mechanics of data ``wrangling''.  By providing
an enabling environment for tools to create and consume
``well-packaged'' data, we can empower these researchers to do more
with less.  In particular, these specifications allow for the
integration of modular, automated data import and validation services
into research data repositories.  We suggest that data quality can
thereby be made ``visible'' by enabling better quality control and
providing standardized visualization options.

\printbibliography

\end{document}
